\documentclass[11pt, a4paper]{article}
\usepackage[utf8]{inputenc}
\usepackage[swedish]{babel}
\usepackage{verbatim}
\usepackage{amsmath}
\usepackage{color, xcolor, enumerate}
\usepackage[noend, linesnumbered]{algorithm2e}
\usepackage{graphicx}
\usepackage{caption}
\usepackage{subcaption}

\bibliographystyle{my-style}

\title{Årsmöte}
\author{Python Sverige}
\date{2013-01-08}

\begin{document}
\maketitle

\section{Mötets öppnande}
Mötet förklarades öppnat 14:00.

\section{Val av mötesordförande}
Fredrik Håård

\section{Val av mötessekreterare}
Frej Connolly

\section{Val av två personer att jämte mötesordföranden justera årsmötets protokoll}
\begin{itemize}
    \item Jyrki Pulliainen
    \item Tome Cvitan
\end{itemize}

\section{Fastställande av föredragningslistan}
Fastställdes. Ty detta är första årsmötet någonsin för föreningen så ströks
följande punkter.

\begin{itemize}
    \item Styrelsens verksamhetsberättelse och ekonomiska berättelse över det senaste året
    \item Revisorernas berättelse
    \item Fråga om fastställande av balansräkning samt disposition av årets resultat
    \item Fråga om ansvarsfrihet för styrelsens ledamöter
    \item Val av valberedning
    \item Framställningar och förslag från styrelsen och från medlemmar som inkommit till styrelsen senast 10 dagar före årsmötet
\end{itemize}

\section{Fastställande av röstlängd}
Samtliga närvarande vid mötet har rösträtt.

\begin{itemize}
    \item Fredrik Håård
    \item Frej Connolly
    \item Jyrki Pulliainen
    \item Laurent Ploix
    \item Nicolas Lara
    \item Rickard Zachrisson
    \item Tome Cvitan
\end{itemize}

\section{Godkännande av kallelse}
Godkändes. Kallelse har gått ut till följande.

\begin{itemize}
    \item pycon-se@python.org
    \item pysthlm-list@meetup.com
    \item conferences@python.org
    \item GothPy (https://gingerhq.com/pycon-se/)
    \item Python SE (https://gingerhq.com/pycon-se/)
\end{itemize}

\section{Beslut om antal ledamöter i styrelsen och ev. antal ersättare}
6 ledamöter och 1 ersättare.

\section{Beslut om man mandatperiod}
1 år.

\section{Beslut om ev ersättning till styrelse m fl.}
Styrelsen arbetar ideellt och får ingen ekonomisk ersättning.

\section{Val av ordförande}
En nominering Fredrik Håård valdes enhälligt till ordförande.

\section{Val av kassör}
En nominering Rickard Zachrisson valdes enhälligt till kassör.

\section{Val av övriga styrelseledamöter och ev. ersättare}
Följande valdes som övriga styrelseledamöter.

\begin{itemize}
    \item Frej Connolly
    \item Jyrki Pulliainen
    \item Nicolas Lara
    \item Tome Cvitan
\end{itemize}
Laurent Ploix valdes som ersättare.

\section{Val av revisorer och ersättare}
Beslut att styrelsen kontaktar en revisor utanför föreningen och styrelsen.

\section{Beslut om eventuella regler och belopp för nästkommande års medlemsavgift}
Föreningens stadgar röstades enhälligt igenom. Se appendix A Stadgar.

Medlemsavgift röstades enhälligt till 0 kr.

\section{Mötets avslutande}
Mötet förklarades avslutat 16:00.

\newpage
\appendix
\section{Stadgar}
\subsection{Namn}
Föreningens namn är Python Sverige.

\subsection{Ändamål}
Föreningen, som är en ideell förening, partipolitiskt och religiöst obunden, har till uppgift att driva PyCon Sverige, en utvecklarkonferens inriktad på programmeringsspråket Python.

Dessutom kan föreningen enligt årsmötets beslut verka för att inkludera Python i andra konferenser i Skandinavien, i övrigt verka för att främja användandet av programmeringsspråket Python, samt stödja annan relaterad verksamhet.

\subsection{Hemvist}
Föreningens skall ha sin hemvist i Stockholms kommun, Stockholms län.

\subsection{Inträde}
Medlemskap i föreningen kan ansökas av enskild person och företag som vill främja föreningens verksamhet och syfte.

\subsection{Avgifter och medlemskap}
Medlem är den som ansöker om medlemskap till föreningen. Ett företag som är medlem räknas som en röst. Medlemsavgiften är 0 kr.

\subsection{Uteslutning}
Bryter medlem mot dessa stadgar, skadar på annat sätt föreningen eller motarbetar dess syften kan styrelsen, om den är enhällig, eller årsmötet utesluta medlemmen ur föreningen.

Den som inte godtar styrelsens beslut om uteslutning, äger hänskjuta frågan till årsmötets prövning.

\subsection{Organ}
Föreningens organ

\begin{itemize}
    \item Årsmöte
    \item Styrelse
    \item Revisorer
\end{itemize}

\subsection{Årsmöte}
Föreningens årsmöte hålls årligen före den 1 juli. Vid årsmötet har varje medlem en röst. Såväl val som övriga frågor avgörs genom öppen omröstning om inte sluten begärs. Votering vid personval skall ske med slutna sedlar. Vid lika röstetal har årsmötets ordförande utslagsröst. Vid personval avgör dock lotten.

Vid ordinarie årsmöte skall följande ärenden behandlas:

\begin{enumerate}
    \item Val av mötesordförande och 
    \item Val av mötessekreterare
    \item Val av två personer att jämte mötesordföranden justera årsmötets protokoll
    \item Fastställande av föredragningslistan
    \item Fastställande av röstlängd
    \item Godkännande av kallelse
    \item Styrelsens verksamhetsberättelse och ekonomiska berättelse över det senaste året
    \item Revisorernas berättelse
    \item Fråga om fastställande av balansräkning samt disposition av årets resultat
    \item Fråga om ansvarsfrihet för styrelsens ledamöter
    \item Beslut om antal ledamöter i styrelsen och ev. antal ersättare
    \item Beslut om mandatperiod
    \item Beslut om ev ersättning till styrelse m.fl.
    \item Val av ordförande
    \item Val av kassör
    \item Val av övriga styrelseledamöter samt ev. ersättare
    \item Val av revisorer och ersättare
    \item Val av valberedning
    \item Beslut om eventuella regler och belopp för nästkommande års medlemsavgift
    \item Framställningar och förslag från styrelsen och från medlemmar som inkommit till styrelsen senast 10 dagar före årsmötet
    \item Vid årsmötet väckta frågor
\end{enumerate}

\subsection{Extra årsmöte}
Extra årsmöte hålls, när styrelsen eller revisorerna finner att det är nödvändigt eller när minst 1/10 av föreningens medlemmar så kräver genom skriftlig begäran till styrelsen.

Av begäran skall framgå det eller de ärenden som medlemmarna vill ha behandlat. På extra årsmöte får endast behandlas de ärenden som angivits i kallelse.

\subsection{Kallelse till årsmöte}
Kallelse till årsmöte skall ske genom kallelse postad till föreningens email-lista senast 30 dagar före stämman.

\subsection{Styrelse}
Föreningens angelägenheter sköts av en styrelse bestående av minst 3 ledamöter och det antal ersättare som årsmötet beslutar.

Mandatperioden kan vara på ett eller två år, enligt årsmötets beslut.

\subsubsection{}
Styrelsen sammanträder när ordföranden finner det lämpligt eller då minst två av styrelsens ledamöter hos ordföranden skriftligen begär sammanträde.

\subsubsection{}
Styrelsen är beslutsför, när de närvarandes antal överstiger hälften av hela antalet ledamöter.

Beslut fattas med enkel majoritet. Vid lika röstetal har ordföranden utslagsröst utom vid personval då lotten avgör.

\subsubsection{}
Styrelsen skall i enlighet med dessa stadgar sköta föreningens angelägenheter.

Styrelsen skall föra en medlemsförteckning. Av förteckningen skall framgå medlemmarnas fullständiga namn och adress.

\subsection{Teckningsrätt}
Styrelsen i sin helhet eller de som styrelsen utser har föreningens teckningsrätt.

\subsection{Räkenskaper}
Föreningens räkenskaper omfattar tiden från den 1 januari till den 31 december.

Styrelsen skall senast 14 dagar före ordinarie årsmöte överlämna sina redovisningshandlingar
till revisorerna.

\subsection{Revisorer}
Föreningens räkenskaper och styrelsens förvaltning skall granskas av minst revisor av ordinarie årsmöte utsedd revisor.

För revisorerna skall utses ersättare. Revisorer och ersättare för dessa utses för ett år. Revisionsberättelse skall lämnas till styrelsen senast 7 dagar före ordinarie årsmötet.

\subsection{Ändring av föreningens stadgar samt likvidation}
Beslut om ändring av dessa stadgar och om föreningens trädande i likvidation fattas på två på varandra följande årsmöten, varav minst en skall vara ordinarie årsmöte. Beslutet skall för att vara gällande på den senare årsmötet ha biträtts av minst 2/3 av de röstande.

\subsection{Upplösning}
Skulle föreningen upplösas skall föreningens behållna tillgångar tillfalla Python Software Foundation http://www.python.org/

\end{document}
